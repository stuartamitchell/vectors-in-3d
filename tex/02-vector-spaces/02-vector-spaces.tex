\documentclass[a4paper,12pt]{amsart}

\usepackage{amsmath}
\usepackage{hyperref}
\usepackage{tikz}
\usepackage{tikz-3dplot}

\title{02 -- Vector spaces}

\begin{document}

    \tdplotsetmaincoords{60}{120}

    \maketitle
    \tableofcontents

    \section{Definition and examples}

    \subsection{Fields}

    A \textbf{field} is a set $F$ together with two binary operations: \textbf{addition} $+: F \times F \to F$ and \textbf{multiplication} $\cdot: F \times F \to F$. Addition and multiplication are bound by the axioms listed below, for all $a, b, c \in F$.
    \begin{enumerate}
        \item Addition and multiplication are associative: $(a + b) + c = a + (b + c)$, and $(a \cdot b) \cdot c = a \cdot (b \cdot c)$.
        \item Addition and multiplication are commutative: $a + b = b + a$, and $a \cdot b = b \cdot a$.
        \item Additive and multiplicative identities: there are two elements $0, 1 \in F$ such that $a + 0 = a$ and $a \cdot 1 = a$.
        \item Additive inverses: For every $a \in F$ there exists an element $(-a)$ such that $a + (-a) = 0$.
        \item Multiplicative inverses: For every $a \neq 0 \in F$ there exists an element $a^{-1}$ such that $a \cdot a^{-1} = a^{-1} \cdot a = 1$.
        \item Distributivity of multiplication over addition: $a \cdot (b + c) = a \cdot b + a \cdot c$.
    \end{enumerate}

    Hopefully, you have observed that these axioms are satisfied by the usual addition and multiplication on the rationals ($\mathbb{Q}$), the reals ($\mathbb{R}$) and the complex numbers ($\mathbb{C}$).

    \subsection{Vector spaces}

    A \textbf{vector space} over a field $F$ is a set $V$ together with two binary operations: 
    
    \begin{description}
        \item[addition] $+: V \times V \to V$, $(\mathbf{u}, \mathbf{v}) \mapsto \mathbf{u} + \mathbf{v}$, and
        \item[scalar multiplication] $\cdot: F \times V \to V$, $(a, \mathbf{v}) \mapsto a \cdot \mathbf{v}$.
    \end{description} 

    The elements of $V$ are called \textbf{vectors}, and the elements of $F$ are called \textbf{scalars}.

    The operations of addition and scalar multiplication satisfy the axioms listed below for all vectors $\mathbf{u}, \mathbf{v}, \mathbf{w} \in V$, and scalars $a, b \in F$.

    \begin{enumerate}
        \item Associativity of addition: $(\mathbf{u} + \mathbf{v}) + \mathbf{w} = \mathbf{u} + (\mathbf{v} + \mathbf{w})$.
        \item Commutativity of addition: $\mathbf{u} + \mathbf{v} = \mathbf{v} + \mathbf{u}$.
        \item Additive identity: There exists an element $\mathbf{0} \in V$ such that $\mathbf{v} + \mathbf{0} = \mathbf{v}$.
        \item Additive inverses: For every $\mathbf{v}$ there exists an element $(-\mathbf{v}) \in V$ such that $\mathbf{v} + (-\mathbf{v}) = \mathbf{0}$.
        \item Compatibility of scalar multiplication with field multiplication: $a \cdot (b \cdot \mathbf{v}) = (ab) \cdot \mathbf{v}$.
        \item Identity element of scalar multiplication: $1 \cdot \mathbf{v} = \mathbf{v}$, where $1 \in F$ is the multiplicative identity of $F$.
        \item Distributivity of scalar multiplication with respect to vector addition: $a \cdot (\mathbf{u} + \mathbf{v}) = a \cdot \mathbf{u} + a \cdot \mathbf{v}$.
        \item Distributivity of scalar multiplication with respect to field addition: $(a + b) \cdot \mathbf{v} = a \cdot \mathbf{v} + b \cdot \mathbf{v}$.
    \end{enumerate}

    The two and three-dimensional vectors with real components described in Lesson 01 form a vector space. They are examples of \textbf{coordinate spaces}. More generally, given a field $F$, all $n$-tuples
    \[ (a_1, a_2, \ldots, a_n) \]
    where $a_1, a_2, \ldots, a_n \in F$ form a vector space over $F$, usually denoted by $F^n$. Addition and scalar multiplication are defined component-wise:
    \[ (a_1, a_2, \ldots, a_n) + (b_1, b_2, \ldots, b_n) = (a_1 + b_1, a_2 + b_2, \ldots, a_n + b_n), \]
    and
    \[ k (a_1, a_2, \ldots a_n ) = (k a_1, k a_2, \ldots, k a_n). \]

    \section{Linear dependence}

    \subsection{Linear combinations}

    A \textbf{linear combination} of vectors $\mathbf{v}_1$, $\mathbf{v}_2$, and  $\mathbf{v}_3$ in a vector space $V$ over a field $F$ is a sum
    \[ k_1 \mathbf{v}_1 + k_2 \mathbf{v}_2 + k_3 \mathbf{v}_3, \]
    where $k_1, k_2, k_3 \in F$.

    For example, let $\mathbf{u} = \begin{pmatrix} 2 \\ 4 \end{pmatrix}$, and $\mathbf{v} = \begin{pmatrix} -1 \\ 3 \end{pmatrix}$. The vector $\begin{pmatrix} 1 \\ 17 \end{pmatrix}$ is the linear combination
    \[ 2 \mathbf{u} + 3 \mathbf{v} = \begin{pmatrix} 4 \\ 8 \end{pmatrix} + \begin{pmatrix} -3 \\ 9 \end{pmatrix}. \]
    
    \subsection{Linear dependence}
    
    A set of vectors $\mathbf{v}_1, \mathbf{v}_2, \ldots, \mathbf{v}_k$ is \textbf{linearly dependent} if one of the vectors can be written as a linear combination of the others, or equivalently, if there exist scalars $a_1, a_2, \ldots, a_k$, not all zero, such that
    \[ a_1 \mathbf{v}_1 + a_2 \mathbf{v}_2 + \cdots + a_k \mathbf{v}_k = \mathbf{0}. \]

    If no vector can be written as a linear combination of the others, or all the scalars in the expression above must be zero, then the vectors are said to be \textbf{linear independent}.

    \subsubsection{Exercise} Show the equivalence of the two statements describing linear dependence above.

    %\section{Basis and dimension}
\end{document}